% This syllabus template was created by:
% Brian R. Hall
% Assistant Professor, Champlain College
% www.brianrhall.net
% And modified extensively by Mark Royer

% Document settings
\documentclass[11pt]{article}
\usepackage[T1]{fontenc}
\usepackage{hyperref}
\usepackage[margin=1in]{geometry}
\usepackage{multirow}
\usepackage[pdftex]{graphicx}
\usepackage{setspace}
\usepackage{svg}
\usepackage{tabularx}
\usepackage{termcal}
\usepackage{xcolor}

\usepackage{lstcustom}

\pagestyle{plain}

\definecolor{turner-blue}{HTML}{001b8f}

\hypersetup{%
  colorlinks=true,
  linkcolor=turner-blue,
  urlcolor=turner-blue
}


% Few useful commands (our classes always meet either on Monday and Wednesday 
% or on Tuesday and Thursday)

\newcommand{\MWClass}{%
\calday[Monday]{\classday} % Monday
\skipday % Tuesday (no class)
\calday[Wednesday]{\classday} % Wednesday
\skipday % Thursday (no class)
\skipday % Friday 
\skipday\skipday % weekend (no class)
}

\newcommand{\MWFClass}{%
\calday[Monday]{\classday} % Monday
\skipday % Tuesday (no class)
\calday[Wednesday]{\classday} % Wednesday
\skipday % Thursday (no class)
\calday[Friday]{\classday} % Friday
\skipday\skipday % weekend (no class)
}

\newcommand{\TRClass}{%
\skipday % Monday (no class)
\calday[Tuesday]{\classday} % Tuesday
\skipday % Wednesday (no class)
\calday[Thursday]{\classday} % Thursday
\skipday % Friday 
\skipday\skipday % weekend (no class)
}

\newcommand{\Holiday}[2]{%
\options{#1}{\noclassday}
\caltext{#1}{#2 (No Class)}
}

\newcommand{\Test}[2]{%
\caltext{#1}{\textbf{#2}}
}

\newcommand{\tf}[1]{%
  \large\textbf{#1}
}

\begin{document}



% This has to be rendered here because \includesvg does not work in
% the \tabularx environment
\newsavebox{\umainelogo}
\begin{lrbox}{\umainelogo}
  \begin{minipage}{3in}
    \includesvg[width=3in]{umaine}
  \end{minipage}
\end{lrbox}

\newsavebox{\cslogo}
\begin{lrbox}{\cslogo}
  \begin{minipage}{1.5in}
    \includesvg[width=1.5in]{csLogo}
  \end{minipage}
\end{lrbox}


{ % Don't indent the table
  \setlength\parindent{0pt}
  \begin{tabularx}{\textwidth}{@{}l X r@{}}
    \begin{minipage}{3in}
      \tf{COS 221}\\      
      \tf{Data Structures in C++} \\
      \tf{MWF, 8-8:50AM, Neville Hall 227} \\\\
    \end{minipage}
    & % White space hfill column
    & {\usebox\umainelogo} \\
    \begin{minipage}{3in}
      \large Mark Royer \\
      \large \href{mailto:mark.royer@maine.edu}{mark.royer@maine.edu} \\
      \large \url{http://cs.umaine.edu/~mroyer} \\
      \large East Annex 223 \& 225 \\
      \large Office Hours: MW 1-4pm \\
      \large (207) 555-1234 \\    
    \end{minipage}
    & % White space hfill column
    & {\usebox\cslogo}
  \end{tabularx}
} % End parindent 0

\subsection*{Note}
\label{sec:note}

This syllabus contains a rough outline of the course and may change in
the future.  If you have any questions, you should check with me.

\section*{Course Description}
\label{sec:desc}
This course covers C++ programming techniques with classes and dynamic
memory management through the study of common data structures such as
lists, stacks, queues, and trees.  Sorting and
searching algorithms are discussed. \\

{ % Don't indent course and textbook info
  \setlength\parindent{0pt}
  \tf{Prerequisite(s):} COS 220 or ECE 177 \\

  \tf{Credit Hours:} 3 \\

  \tf{Text:} \emph{Introduction to Programming with C++}, 3\textsuperscript{rd} Edition

  \tf{Author:} Y. Daniel Liang

  \tf{ISBN-13:} 978-0-13-325281-1
} % End course and textbook info

\subsection*{Grade Distribution:}
\label{sec:grades}

\begin{center}
  \begin{tabular}{ l l }
    Class Participation & 5\% \\
    Assignments & 35\% \\
    Quizzes(2)  & 10\% \\
    Midterm Exams(2)  & 20\% \\
    Final Exam  & 30\%
  \end{tabular}
\end{center}

\subsection*{Letter Grade Distribution:}
\label{sec:lettergrades}

\begin{center}
  \begin{tabular}{ l l | l l }
    \textgreater= 93.00 & A & 73.00 - 76.99 & C \\
    90.00 - 92.99 & A-  & 70.00 - 72.99 & C- \\
    87.00 - 89.99 & B+  & 67.00 - 69.99 & D+ \\
    83.00 - 86.99 & B  & 63.00 - 66.99 & D \\
    80.00 - 82.99 & B-  & 60.00 - 62.99 & D- \\
    77.00 - 79.99 & C+  & \textless= 59.99 & F
  \end{tabular}  
\end{center}

\section*{Course Policies:}
\label{sec:policies}

\begin{itemize}
\item \textbf {General:} \textbf {No makeup} quizzes or exams will be
  given.
\item \textbf {Assignments}
  \begin{itemize}
  \item Students are expected to work independently. \textbf{Offering}
    and \textbf{accepting} solutions from others is an act of
    \textbf{plagiarism}, which is a serious offense and \textbf{all
      involved parties will be penalized according to the Academic Honesty
      Policy}. Discussion among students is encouraged, but when in doubt,
    ask me.
  \item \textbf{Late assignments will lose 30\% every week late.
      Students must hand in every assignment}.
  \end{itemize}
\item \textbf{Attendance and Absences}
  \begin{itemize}
  \item Students are responsible for all missed work, regardless of
    the reason for absence.
  \item It is also the absentee's responsibility to get all missing
    notes or materials.
  \end{itemize}
\end{itemize}

\section*{Academic Honesty Policy Summary:}
\label{sec:honesty}
% This should be specific to your instituition, an example is provided.

\subsection*{Introduction}
\label{sec:honesty-intro}

In addition to skills and knowledge, The University of Maine aims to
teach students appropriate Ethical and Professional Standards of
Conduct. The Academic Honesty Policy exists to inform students and
Faculty of their obligations in upholding the highest standards of
professional and ethical integrity. All student work is subject to the
Academic Honesty Policy. Professional and Academic practice provides
guidance about how to properly cite, reference, and attribute the
intellectual property of others. Any attempt to deceive a faculty
member or to help another student to do so will be considered a
violation of this standard.

\subsection*{Instructor's Intended Purpose}
\label{sec:honesty-purp}

The student's work must match the intended purpose for an
assignment. Each student must clarify outstanding questions of that
intent for a given assignment.

\subsection*{Unauthorized/Excessive Assistance}
\label{sec:honest-assist}

The student may not give or get any unauthorized or excessive
assistance in the preparation of any work.

\subsection*{Authorship}
\label{sec:honesty-auth}

The student must clearly establish authorship of a work. Referenced
work must be clearly documented, cited, and attributed, regardless of
media or distribution. Even in the case of work licensed as public
domain or Copyleft, (See: \url{http://creativecommons.org/}) the student
must provide attribution of that work in order to uphold the standards
of intent and authorship.

\subsection*{Declaration}
\label{sec:honesty-dec}

Online submission of, or placing one's name on an exam, assignment, or
any course document is a statement of academic honor that the student
has not received or given inappropriate assistance in completing it
and that the student has complied with the Academic Honesty Policy in
that work.

The full text of the Academic Honesty Policy is in the \emph{Student
  Handbook}.

\section*{Tentative Course Outline:}
\label{sec:outline}

\begin{enumerate}
\item Programming environment/tools, Ubuntu VirtualBox image, gEdit, Makefile/gcc/make, DDD
\item Pointers and Memory (\lstinline$new$/\lstinline$delete$),
  Chapter 11 (Pointers and Dynamic Memory)
\item Classes/Objects/Constructors and Destructors, Chapter 9 (Objects
  and Classes), Chapter 10 (Object-Oriented Thinking)
\item Dynamic Data Structures (lists/stacks/queues/trees), Chapter 12
  (Templates, Vectors and Stacks), Chapter 20 (Linked Lists, Queues
  and Priority Queues), Chapter 21 (Binary Search Trees)
\item File IO (text/binary), \lstinline$fstream, <<, >>, endl$, Chapter 13.1-13.6
\item Sorting and Searching (algorithm efficiency), Chapter 19
  (Sorting), Chapter 21 (Binary Search Trees)
\item STL/Templates, Chapter 12 (Templates, Vectors and Stacks)
\item Inheritance, Chapter 15 (Inheritance and Polymorphism)
\item Bitwise operators/ASCII Codes \lstinline$&, |, ^, ~, <<, >>, >>>$
\item Hashing
\end{enumerate}

Class dates with quizzes and tests are shown below.  These dates are
unlikely to change.  The details of what will be covered during other
classes will be filled out as the class progresses.


\section*{Tentative Schedule:}
\label{sec:sched}


\renewcommand{\calprintdate}{%
  \ifdefined\HCode
  {\HCode{<span class="caldate">}}
  \ifnewmonth\framebox{\textbf{\monthname\ \ordinaldate}}%
  \else \textbf{\ordinaldate}\fi
  {\HCode{</span>} }
  \else
  \ifnewmonth\framebox{\textbf{\monthname\ \ordinaldate}}%
  \else \textbf{\ordinaldate}\fi
  \fi
}

\renewcommand{\calprintclass}{%
  \ifdefined\HCode
  {\HCode{<span class="calday">}}
  \textbf{\small\theclassnum}
  {\HCode{</span>} }
  \else
  \textbf{\small\theclassnum}
  \fi
}

\begin{center}
\begin{calendar}{9/2/2013}{16} % Semester starts on 9/3/2013 and last for 16
                    % weeks, including finals week
\setlength{\calboxdepth}{.3in}
\MWFClass
% schedule
%\caltexton{1}{Labor Day - No class} DON'T NEED THIS HOLIDAYS DECLARED BELOW
\caltexton{1}{Review}
\caltextnext{Makefile,Doxygen,DDD}
\caltextnext{Introduction to pointers 11.1 - 11.6}
\caltextnext{More memory 11.7 - 11.9}
\caltextnext{Classes 9.1 - 9.7}
\caltextnext{Classes 9.8-9.11, Dynamic 11.10-11.15}
\caltextnext{OO Thinking 10.1-10.6}
\caltextnext{Class Design 10.7-10.10}
\caltextnext{Intro To Templates 12.1-12.5}
\caltextnext{Templates 12.6-12.10}
\caltextnext{Stacks,Vector 12.6-12.10}
\caltextnext{LinkedList 20.1-20.4}
\caltextnext{Recursion 17.1-17.4}
\caltextnext{Recursion 17.5-17.8}
\caltextnext{Analyzing Algorithms 18.1-18.3}
\caltextnext{Analyzing Algorithms 18.4-18.7}
\caltexton{18}{Exam Review}
\caltextnext{Operator overloading 14.1-14.12}
\caltextnext{Analyzing Algorithms}
\caltextnext{Sorting 19.1-19.4}
\caltextnext{Sorting 19.5-19.6}
\caltextnext{Sorting 19.7}
\caltextnext{File I/O 13.1-13.6}
\caltextnext{File I/O 13.7-13.9}
\caltextnext{Sorting 19.8}
\caltextnext{Iterators, Queues 20.5-20.8}
\caltextnext{Priority Queues 20.9}
\caltextnext{Inheritance 15.1-15.3}
\caltextnext{Inheritance 15.4-15.10}
\caltextnext{Binary Search Trees 21.1-21.2}
\caltextnext{AVL Trees 26.1-26.4}
\caltextnext{AVL Trees 26.5-26.9}
\caltextnext{Review}
\caltexton{36}{Finish AVL Trees}
\caltextnext{Splay Trees 26.10}
\caltextnext{More splay trees}
\caltextnext{Graphs 24.1-24.3}

\caltextnext{Graphs 24.4-24.7}
\caltextnext{Review}

% ... and so on

% Tests and quizes
\Test{9/20/2013}{Quiz 1}
\Test{10/11/2013}{Exam 1}
\Test{11/1/2013}{Quiz 2}
\Test{11/25/2013}{Exam 2}

% Important dates
\caltext{9/9/2013}{Last day to add class}
\caltext{9/16/2013}{Last day to drop class}
\caltext{11/15/2013}{Last day to withdraw}

% Holidays
\Holiday{9/2/2013}{Labor Day}
\Holiday{10/14/2013}{Fall break}
\Holiday{11/27/2013}{Thanksgiving}
\Holiday{11/29/2013}{Thanksgiving}
% ... and so on

\options{12/16/2013}{\noclassday} % finals week
\options{12/17/2013}{\noclassday} % finals week
\options{12/18/2013}{\noclassday} % finals week
\options{12/19/2013}{\noclassday} % finals week
\options{12/20/2013}{\noclassday} % finals week

\Test{12/16/2013}{Final Exam 9:30 AM}
\end{calendar}
\end{center}

\end{document}




%  LocalWords:  MWF Neville ECE Liang VirtualBox gEdit DDD fstream OO
%  LocalWords:  Destructors endl STL Polymorphism Bitwise Doxygen AVL
%  LocalWords:  LinkedList
